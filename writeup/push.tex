% Created 2021-01-21 Thu 20:09
% Intended LaTeX compiler: pdflatex
\documentclass[11pt]{article}
\usepackage[utf8]{inputenc}
\usepackage[T1]{fontenc}
\usepackage{graphicx}
\usepackage{grffile}
\usepackage{longtable}
\usepackage{wrapfig}
\usepackage{rotating}
\usepackage[normalem]{ulem}
\usepackage{amsmath}
\usepackage{textcomp}
\usepackage{amssymb}
\usepackage{capt-of}
\usepackage{hyperref}
\setlength{\parindent}{0pt}
\author{Max Fan}
\date{\today}
\title{On the Game Push}
\hypersetup{
 pdfauthor={Max Fan},
 pdftitle={On the Game Push},
 pdfkeywords={},
 pdfsubject={},
 pdfcreator={Emacs 27.1 (Org mode 9.5)}, 
 pdflang={English}}
\begin{document}

\maketitle
For the following lemmas, I'm considering all games in which left has a winning strategy (values > 0).
By swapping \underline{L} and \underline{R}, you can get the same results for \underline{R} and thus for all valid positions of the game.

\bigskip

Notation:

\begin{itemize}
\item Each literal symbol will be underlined (\underline{L} or \underline{R})

\item \(\mathbb{X}_{n}\) represents, without loss of generality any sequence of characters or spaces of length \(n\).
\(\mathbb{X}_{n}'\) represents the sequence \(\mathbb{X}\) with the rightmost space removed, if one exists.
If a rightmost space does not exist, then \(\mathbb{X}_{n}=\mathbb{X}_{n}'\)

\item Upper case symbols (like \(G\), \(A\), and \(B\)) represent sequences/game states.

\item Greek symbols will be used to denote a single character, that can either be \(\underline{L}\), \(\underline{R}\),  or \(\_\).

\item The entire game state will be completely underlined (e.g. \(\underline{\underline{L} \underline{L} \underline{R}}\))

\item ``+'' means concatenation of a sequence
\end{itemize}

The evaluation of a sequence will be denoted like so, where \(G\) is the current game state and the number preceding the colon is the square being pushed:
$$G := \underline{\underline{L} \underline{L} \underline{R}}$$
$$G_{0}: \underline{\underline{L} \underline{L} \underline{R}} \to \underline{\_ \underline{L} \underline{R}}$$
$$G_{1}: \underline{\underline{L} \underline{L} \underline{R}} \to \underline{\underline{L} \underline{R}}$$
$$G_{2}: \underline{\underline{L} \underline{L} \underline{R}} \to \underline{\underline{L} \underline{R}}$$

The game is zero-indexed and always ends in \(\underline{R}\) or \(\underline{L}\).

Different game positions can be compared like so:
$$G > G_{1} \ge G_{2}$$

\section*{Lemmas}
\label{sec:org2a07434}
\subsection*{Left Accessibility Axiom (done)}
\label{sec:org0fd5709}
If one game position \(Q\) is accessible from another game position \(P\), ending in \(\underline{L}\), by a series of moves, then \(P > Q\).

Rationale:
Whenever a move is played, by \(\underline{L}\) or \(\underline{R}\), the value of the game is always closer to zero.

\subsection*{Right Accessibility Axiom (done)}
\label{sec:orgcfcd220}
If one game position \(Q\) is accessible from another game position \(P\), ending in \(\underline{R}\), by a series of moves, then \(P < Q\).

Rationale:
Whenever a move is played, by \(\underline{L}\) or \(\underline{R}\), the value of the game is always closer to zero.

\subsection*{Negation Axiom (done)}
\label{sec:org0477d7d}
Swapping all values of \(\underline{L}\) and \(\underline{R}\) in the game state is logically equivalent to negating the value of the game.

\subsection*{Left Reduction Lemma (done)}
\label{sec:org2324387}
If \(\mathbb{X}_{n}\) has spaces and ends in \(\underline{L}\), then:
$$\mathbb{X}_{n} > \mathbb{X}_{n}'$$
If \(\mathbb{X}_{n}\) has no spaces and ends in \(\underline{L}\), then:
$$\mathbb{X}_{n} = \mathbb{X}_{n}'$$

\subsection*{Right Reduction Lemma (done)}
\label{sec:org178002c}
If \(\mathbb{X}_{n}\) has spaces and ends in \(\underline{R}\), then:
$$\mathbb{X}_{n} < \mathbb{X}_{n}'$$
If \(\mathbb{X}_{n}\) has no spaces and ends in \(\underline{R}\), then:
$$\mathbb{X}_{n} = \mathbb{X}_{n}'$$

\subsection*{Same-Length Concat Lemma}
\label{sec:org78a6869}
If \(P > Q\) and \(P\) and \(Q\) are of the same length and \(A\) ends in \(\underline{L}\), then \(P + A > Q + A\).
Proof.

\bigskip
Starting from the rightmost square, if \(P > Q\) then \(P \not= Q\) and \(P\) and \(Q\) must differ at a location \(n\).
Let \(P = P_{1} + \underline{\alpha} + A\) and \(Q = Q_{1} + \underline{\beta} + A\) where \(\beta\) is the symbol at the location \(n\) where \(P\) and \(Q\) differ.
By construction, \(\underline{\alpha} \not= \underline{\beta}\).
Suppose for contradiction that \(\underline{\alpha} < \underline{\beta}\).
Then,

Suppose \(P\) is positive.
\(Q\) is either positive, zero, or negative.
If zero, done .

\subsection*{General Prepend Lemma (done)}
\label{sec:org16f51a2}
If \(A\) ends in \(\underline{L}\), then for any characters \(\underline{\alpha}\) and \(\underline{\beta}\) such that \(\underline{\alpha} > \underline{\beta}\), then \(P := \underline{\underline{\alpha} + A} > Q := \underline{\underline{\beta} + A}\). Proof.

There are six cases to examine since \(\underline{\alpha}\) and \(\underline{\beta}\) can either be
\(\underline{L}\), \(\underline{R}\), \(\_\):
\begin{itemize}
\item \(\underline{\alpha} = \underline{L}\), \(\underline{\beta} = \underline{\_}\)
In this case, \(P_{0}: \underline{\underline{\alpha} + A} \to \underline{\underline{\beta} + A}\), so
\(P > Q\) by the left accessibility axiom.

\item \(\underline{\alpha} = \underline{\_}\), \(\underline{\beta} = \underline{R}\)
In this case, we want to show that \(P > Q\).
It is equivalent to show \(-Q > -P\).
By the negation axiom, \(-Q\) and -\(P\) both end in \(\underline{R}\).
\(-Q\) and \(-P\) only differ at the zeroth place.
\$\$
\end{itemize}


Previous proof of this point:
In this case, \(Q\) has more \(\underline{R}\) values than \(P\) and the same number and location of \(\underline{L}\) values.
Since right has more moves available in \(Q\) than in \(P\), \(P > Q\) (TODO: better justification).

\begin{itemize}
\item \(\underline{\alpha} = \underline{L}\), \(\underline{\beta} = \underline{R}\)
From the first case and the second case and the transitive property,
\(P > Q\) follows.
\end{itemize}

In all three cases, \(P > Q\), so necessarily, \(P > Q\).

\subsection*{Append Lemma (unproven)}
\label{sec:org137b120}
For any sequence of characters (empty or not), \(\mathbb{X}_{n}\):
$$\underline{\mathbb{X}_{n} \underline{L}} > \underline{\mathbb{X}_{n}}$$

\subsection*{Prepend Lemma (WIP)}
\label{sec:org9c5a268}
For any sequence of characters (empty or not), \(\mathbb{X}_{n}\):
$$P := \underline{\_ \mathbb{X}_{n} \underline{L}} > Q := \underline{\mathbb{X}_{n} \underline{L}}$$

Proof.

\bigskip

Suppose that \(\mathbb{X}_{n}\) does not contain spaces.
Then,
$$P_{n} : \underline{\_ \mathbb{X}_{n} \underline{L}} \to \underline{\mathbb{X}_{n} \underline{L}}$$
So, \(P_{n} = Q\) and \(Q\) is accessible from \(P\).
Therefore, by the left accessibility axiom, \(P > Q\).

\bigskip

Suppose that \(\mathbb{X}_{n}\) does contain spaces.
Then, \(\underline{\mathbb{X}_{n} \underline{L}} \to \underline{\mathbb{X}_{n}' \underline{L}}\).
Without loss of generality, split \(\mathbb{X}_{n}\) by its rightmost space.
Define \(\mathbb{X}_{n} = A + \underline{\_} + B\) where \(\mathbb{X}_{b}\) contains no spaces.
Then, \(\mathbb{X}_{n}' = A + B\)

\bigskip
\(\mathbb{X}_{n}\) either contains or does not contain spaces, so by disjunction elimination, the lemma holds.

Scratch (problematic because it depends on the concat lemma):
Suppose that \(\mathbb{X}_{n}\) does contain spaces.
Then, \(\underline{\mathbb{X}_{n} \underline{L}} \to \underline{\mathbb{X}_{n}' \underline{L}}\).
By the left reduction lemma, since the sequence \(\mathbb{X}_{n}\) has spaces, \(\mathbb{X}_{n} > \mathbb{X}_{n}'\).
\subsection*{The Fundamental Theorem of Push}
\label{sec:orge864b6a}
Statement: For any given game, pushing the leftmost piece is always the optimal move.

\bigskip

A valid game ends in either \(\underline{L}\) or \(\underline{R}\).
We first induct over all games that end in \(\underline{L}\).

\bigskip

Base case:
Pushing the first \(\underline{L}\) is the optimal move in the game: \(\underline{\underline{L}\underline{L}}\).

Inductive hypothesis:
Given any sequence of character(s) \(\mathbb{X}\), the most

\bigskip



\subsection*{Acknowledgements}
\label{sec:orgc80ed16}
Sophie Vulpe provided excellent feedback and encouragement.
\end{document}
